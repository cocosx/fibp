\documentclass[11pt,draft,oneside]{fithesis}
\usepackage[utf8]{inputenc}
\usepackage[IL2]{fontenc}
\usepackage[plainpages=false, pdfpagelabels]{hyperref}
\thesistitle{Design and implementation of a social network for making acquaintances}
\thesissubtitle{Bachelor thesis}
\thesisstudent{Marek Bryša}
\thesiswoman{false}
\thesisfaculty{fi}
\thesislang{en}
\thesisyear{jaro 2012}
\thesisadvisor{doc. Ing. Michal Brandejs, CSc.}

\begin{document}
\FrontMatter
\ThesisTitlePage

\begin{ThesisDeclaration}
\DeclarationText
\AdvisorName
\end{ThesisDeclaration}

\begin{ThesisThanks}
Zde bude uvedeno... 
\end{ThesisThanks}

Obdobně jako poděkování se mohou vysadit shrnutí a klíčová 
slova pomocí prostředí Thesisacti a ThesisKeyWordsi.

\MainMatter
\tableofcontents
\chapter*{Introduction}
\chapter{Design}
\section{Existing social networks for making acquaintances}
\section{User data privacy}
\section{Conclusion}

\chapter{Implementation}
\section{Technologies}
\section{Basic functionality}
	\subsection{User registration}
	\subsection{Profile photo upload}
	\subsection{Acquantaince selection}
	\subsection{Notifications}
	\subsection{Chat}
\section{Implementation in detail}
	\subsection{Security}
	\subsection{I18n}
	\subsection{Geolocation}
	\subsection{Graphical design}
\chapter{Conclusion}

% Následují další kapitoly a podkapitoly, popřípadě závěr, dodatky, 
% seznam literatury či použitých obrázků nebo tabulek.

\bibliographystyle{plain}  % bibliografický styl 
\bibliography{bp.bib} % soubor s citovanými
                           % položkami bibliografie 

\end{document}